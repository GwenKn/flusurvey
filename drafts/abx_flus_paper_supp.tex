\documentclass{article}
	\usepackage{amsmath}
	\usepackage[english]{babel}
	\usepackage{breakcites}
	\usepackage[font=small,labelfont=bf]{caption}
	\usepackage{fullpage}
	\usepackage[]{geometry}
	\usepackage[]{graphicx}
	\usepackage{palatino}
	\usepackage{rotating}
	\usepackage{subcaption}
	\usepackage{tikz}
	\usepackage{csvsimple}
	\usepackage{multirow}
	\usepackage{verbatim}
	\bibliographystyle{unsrt}

\title{Supplementary for: Antibiotic usage in Flusurvey}

\author{Gwen, Seb (John? Ceire? )} 

	\date{}
	
	\begin{document}
	 
	\maketitle{}

	\tableofcontents{}

	\clearpage

	\section{Data cleaning}

	Initially, 28,332 unique episodes were recorded. To clean this data, several variables where ambiguous answers were recorded were removed:

	\begin{itemize}
		\item Visit: only those episodes where the participant had responded to the medical visit question were kept for analysis (no visit, appointment made or specified visit type). This removes 380 episodes. 
		\item Health score: only those episodes with a finite health score, between 0 and 100, were kept. This removes 13 episodes. Those episodes with infinite or missing health scores had their health score set to be "NA". 
		\item Vaccine status: any episode where the participant had responded with a "don't know" to whether they had been vaccinated that year was removed. This removed 35 episodes. 
		\item ILI and fever: if an NA was entered for this status then the episode was removed. This removes 30 episodes. 
	\end{itemize}

	A total of 27,874 episodes were analysed from 3,654 participants. The total number of episodes over the 6 seasons per participant ranged from 1 to 22, with a mean of 6 episodes per participant. 

	\section{Univariate analysis}

\subsection{Compare seasons}

Above 3,000 episodes were recorded for each season, ranging from a total of 3,093 - 5,647 episodes per season (Figure \ref{fig:compareseasons}). The number of antibiotic prescriptions per season ranged from 116 - 271, suggesting that analysis of prescriptions per year may be inappropriate (Figure \ref{fig:compareseasons}). The overall prescription rate was similar for each season at around 4\% (Figure \ref{fig:compareseasons}). This equates to participants receiving antibiotics in ~4\% of the episodes of illness. 

	\begin{figure}[htbp]
		\centering
		\includegraphics[width=1\textwidth]{../plots/compare_seasons.pdf}
		\caption{Comparing baseline data across seasons}
		\label{fig:compareseasons}
	\end{figure} 

\begin{itemize}
	\item don't include season - not much difference and too little data
\end{itemize}

\subsection{By age and season}

Antibiotic prescription rates are higher for children (age \textless 18yrs) and the elderly ( \textgreater 65 yrs) (Figure \ref{fig:ageseason}). This is consistent with previous data. The same pattern can be seen across the seasons (Figure \ref{fig:ageseason}). 

	\begin{figure}[htbp]
		\centering
		\includegraphics[width=1\textwidth]{../plots/age_season_antibiotic_prescription_rate.pdf}
		\caption{Comparing antibiotic prescription rate by age and season}
		\label{fig:ageseason}
	\end{figure} 

\begin{itemize}
	\item need to include age
\end{itemize}

\subsection{Influenza like illness}

Episodes could be split into those where participants reported symptoms that, under the ECDC standards [ref], could be defined as an Influenza like illness (ILI) or not. Moreover, there was the option to report if a participant suffered fever during an episode. Considering antibiotic prescriptions by ILI (Figure \ref{fig:ilif}), shows that episodes where participants experienced ILI were much more likely to receive antibiotics than those without ILI: 3.5\% vs. 7.8\% (Figure \ref{fig:ili}). The addition of fever to the symptoms only increased the percentage of episodes that involved an antibiotic prescription: 3.9\% vs. 11\% (Figure \ref{fig:ili.fever}).

\begin{figure}[htbp]
    \centering
    \begin{subfigure}[b]{0.4\textwidth}
   		\includegraphics[width=1\textwidth]{../plots/ili.pdf}
   		\caption{}
   		\label{fig:ili}
    \end{subfigure}
    \begin{subfigure}[b]{0.4\textwidth}
        \includegraphics[width=1\textwidth]{../plots/ili_fever.pdf}
		\caption{}
		\label{fig:ili.fever}
    \end{subfigure}
    \caption{Antibiotic prescriptions split by those with ECDC standards defined Influenza like illness (ILI) (a) or an ILI and fever (b)}
    \label{fig:ilif}
\end{figure}

\begin{itemize}
	\item don't include ili
	\item include ili.fever (SF: tracks 'flu season well)
\end{itemize}

\subsection{Region}

Analysis of prescription rate by region shows little variation from the approximately 4\% mean rate per episode (Figure \ref{fig:region}). A slightly higher mean is seen for Northern Ireland. The Channel Islands and the Isle of Man were excluded for having too few reports (N = 23 or 7 respectively). 

	\begin{figure}[htbp]
		\centering
		\includegraphics[width=1\textwidth]{../plots/Region_results_(bigcount_nolabel).pdf}
		\caption{Comparing antibiotic prescription rate by region. Here N is the number of episodes analysed.}
		\label{fig:region}
	\end{figure} 

\begin{itemize}
	\item don't include region - not much difference and too little data
\end{itemize}

\subsection{Influenza vaccine status}

Participants were asked to state whether they had received the influenza vaccine that year. 35 reported that they "did not know" and were excluded from the analysis. Grouping by vaccine status that year, revealed that those who had been vaccinated were more likely to receive antibiotics per episode (3.5\% vs. 5.2\%) (Figure \ref{fig:vx}). This may be due to increased access to care for those who are vaccinated. 

If we consider prescription rate by vaccine status and age (Figure \ref{fig:vx.age}), it can be seen that those in the risk factor age for vaccination (children and elderly), do not have a difference in prescription rate by vaccine status. However, adults (18-65yo) are more likely to receive antibiotics if they have had the influenza vaccine suggesting that there is a link to care behaviour. 

\begin{figure}[htbp]
    \centering
    \begin{subfigure}[b]{0.45\textwidth}
   		\includegraphics[width=1\textwidth]{../plots/vaccine_(noyesonly).pdf}
   		\caption{}
   		\label{fig:vx}
    \end{subfigure}
    \begin{subfigure}[b]{0.45\textwidth}
        \includegraphics[width=1\textwidth]{../plots/vaccine_age_(noyesonly).pdf}
		\caption{}
		\label{fig:vx.age}
    \end{subfigure}
    \caption{Antibiotic prescriptions split by those who had been vaccinated that year across all age groups (a) and then split by age group (b)}
    \label{fig:vxf}
\end{figure}

\begin{itemize}
	\item include vaccine status, but needs age to be taken into account
\end{itemize}


\subsection{Healthcare visit during episode}

Participants were asked if they had visited any medical services due to their symptoms (Figure \ref{fig:healthss}). All visits in a single episode were grouped together.

If we consider antibiotic prescription rate by any medical service visit (Figure \ref{fig:visityn}), we can see that rates are much higher if a visit to a medical service was recorded (4\% vs. 40\%). This is as would be expected as antibiotics should only be available by prescription in the UK. 

Looking at the type of medical service visits reveals little significant difference in rates. Importantly for analysis, most had few episodes linked to this type of medical visit behaviour. Considering only those that had \textgreater 100 episodes ("*" indication, Figure \ref{fig:visit}), suggests that higher prescription rates were found for those who visited a GP or the Hospital than visiting "Other" medical services. 

	\begin{figure}[htbp]
		\centering
		\includegraphics[width=1\textwidth]{../plots/Visit_results_yn.pdf}
		\caption{Did a participant visit a medical service during this episode?}
		\label{fig:visityn}
	\end{figure} 

\begin{figure}[htbp]
    \centering
    \begin{subfigure}[b]{0.35\textwidth}
   		\includegraphics[width=1\textwidth]{../plots/health_visit_screenshot.jpg}
   		\caption{}
   		\label{fig:healthss}
    \end{subfigure}
    \begin{subfigure}[b]{0.55\textwidth}
        \includegraphics[width=1\textwidth]{../plots/Visit_results.pdf}
		\caption{}
		\label{fig:visit}
    \end{subfigure}
    \caption{The questions asked of participants (a) and the results for antibiotic prescription rate (b). In (b), those on the left of the first dashed line did not visit a medical service during the episode ("None"). Those in the centre section only reported visiting one medical service, to the right multiple services in an episode. A "*" indicates that more than 100 episodes had this visit behaviour recorded.}
    \label{fig:visitf}
\end{figure}


\begin{itemize}
	\item Don't include the individual health visit behaviours
	\item convert to visit healthcare or not variable = visit.medical.service.no
\end{itemize}

\subsection{Higher education and main activity}

Analysing prescription rates by highest education status (Figure \ref{fig:hedu}) or main daily activity (Figure \ref{fig:maina}), shows that these may have a significant impact. Those with lower levels of education ("gcse") vs. the highest ("msc"), had large differences in rates (8.6\% vs. 3.4\% respectively) (Figure \ref{fig:hedu}). This is consistent with previous reports of antibiotic exposure being higher in children in mothers who have lower education levels [ref]. Age may account for some of the trends by main activity as well as underlying health (Figure \ref{fig:maina}), with those on long term leave and retired having high levels of antibiotic prescriptions. 

\begin{figure}[htbp]
    \centering
    \begin{subfigure}[b]{0.48\textwidth}
   		\includegraphics[width=1\textwidth]{../plots/highest_education.pdf}
   		\caption{}
   		\label{fig:hedu}
    \end{subfigure}
    \begin{subfigure}[b]{0.48\textwidth}
        \includegraphics[width=1\textwidth]{../plots/main_activity.pdf}
		\caption{}
		\label{fig:maina}
    \end{subfigure}
    \caption{Prescription rates by highest education status (a) and main activity (b)}
    \label{fig:sociof}
\end{figure}

\begin{itemize}
	\item Don't include highest education as 19,848 (71\%) of episodes have no information on this
	\item Include main activity - affected by age and risk? All episodes have this data.
\end{itemize}

\subsection{Gender}

Women are more likely to be prescribed antibiotics due to the frequency of urinary tract infections (UTIs) [ref]. This trend is seen in this data (though not significantly) (Figure \ref{fig:gend}), unless age is factored into the analysis where it is seen for young adults (18-45yo) (Figure \ref{fig:genda}). 

\begin{figure}[htbp]
    \centering
    \begin{subfigure}[b]{0.48\textwidth}
   		\includegraphics[width=1\textwidth]{../plots/Gender_results.pdf}
   		\caption{}
   		\label{fig:gend}
    \end{subfigure}
    \begin{subfigure}[b]{0.48\textwidth}
        \includegraphics[width=1\textwidth]{../plots/Gender_age_results.pdf}
		\caption{}
		\label{fig:genda}
    \end{subfigure}
    \caption{Prescription rates by gender: alone (a) and by age (b)}
    \label{fig:gendf}
\end{figure}

\begin{itemize}
	\item Include gender (and age)
\end{itemize}

\subsection{Contact with children}

Those with frequent contact with children might be expected to be ill more often and hence may receive antibiotics more often. This is supported by this data (Figure \ref{fig:childf}). 

	\begin{figure}[htbp]
		\centering
		\includegraphics[width=0.8\textwidth]{../plots/freq_contact_children.pdf}
		\caption{Comparing antibiotic prescription rate by whether you have "frequent contact with more than 10 children". }
		\label{fig:childf}
	\end{figure} 

	\begin{itemize}
	\item Include frequent contact with children
\end{itemize}

\subsection{Health score}

Health score is a self-reported, weekly value between 0 and 100. The baseline health score is the median health score (?) when a participant is not ill. This changes by season. A minimum health score is calculated as the minimum per episode. Here a normalised health score was analysed where the difference between the minimum and baseline was divided by the baseline health score. This then gives "-1" when the minimum health score during an episode was 0 (the worst it could be). 

Health score is not recorded well. Out of the 27,939 episodes, 45\% (12,647 episodes) have no reported minimum or baseline health score. These were removed for this analysis. Then plotting whether a participant received antibiotic against normalised health score showed little correlation (Figure \ref{fig:hsf}). We might have expected lower health score to be linked to more likely to receive antibiotics. 

	\begin{figure}[htbp]
		\centering
		\includegraphics[width=0.8\textwidth]{../plots/health_score_yn_all.pdf}
		\caption{Antibiotic prescription by health score}
		\label{fig:hsf}
	\end{figure} 

\begin{itemize}
	\item Don't include health score
\end{itemize}

\subsection{Underlying health problem}

Flusurvey participants are asked if they have an underlying health issue that may affect their risk of becoming ill. This included diabetes and asthma. Of those episodes where antibiotics were prescribed, 5,248 were from participants who had an underlying health issue. 

It might be expected that those with an underlying health problem would also have a higher risk of receiving antibiotics. This is what we found in the Flusurvey data (Figure \ref{fig:risk}). 

	\begin{figure}[htbp]
		\centering
		\includegraphics[width=0.8\textwidth]{../plots/risk.pdf}
		\caption{Antibiotic prescription by whether the participant had an underlying health problem}
		\label{fig:risk}
	\end{figure} 

\begin{itemize}
	\item Include whether a participant has an underlying health issue but not the details of the types of health score as unclear which would be more linked to antibiotic use and becomes too complex (not enough data)
\end{itemize}

\clearpage 





\section{Multivariate analysis?}

Based on the above univariate analysis, the following were included in the multivariate analysis: 

\begin{itemize}
	\item Age (will be regularised) [age]
	\item If have ILI and fever [ili.fever]
	\item Vaccine status [vaccine.this.year]
	\item Whether they visited a medical service or not [Visit.medical.service.no]
	\item Main activity [main.activity]
	\item Gender [gender]
	\item Frequent contact with children [frequent.contact.children]
	\item Underlying health risk [norisk]
\end{itemize}



\section{Thoughts}

\begin{itemize}
	\item should we include frequent contact with elderly as well as with children? 
\end{itemize}


% \bibliography{../flusurvey}

	\end{document}